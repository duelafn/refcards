\documentclass[letterpaper]{article}
\usepackage[centering,width=7.25in,height=9.75in,includefoot]{geometry}
\usepackage[parfill]{parskip}
\usepackage{amsmath,amssymb,array,theorem,graphicx}
\usepackage{multicol,xcolor,shortvrb,fancyvrb}
\usepackage[user,titleref]{zref}
\usepackage{xspace}
\usepackage[breaklinks]{hyperref}
\hypersetup{
  pdftitle={OpenPGP + GnuPG Tutorial and Reference Guide},
%   pdfsubject={Further (short) Description of Document},
  pdfauthor={Dean Serenevy},
}
\renewcommand{\title}[2][]{\begin{center}\sffamily\bfseries\Large #2\ifx\relax#1\else\\[1ex]\normalsize #1\fi\end{center}\vspace{1ex}}
\let\epsilon\varepsilon\let\phi\varphi
\makeatletter
\def\section{\@startsection{section}{1}{\z@}{-2ex\@plus-.5ex\@minus-.2ex}{.001ex\@plus0ex\@minus.3ex}{\normalfont\Large\sffamily\bfseries}}
\def\subsection{\@startsection{subsection}{2}{\z@}{-2ex\@plus-.5ex\@minus-.2ex}{.001ex\@plus0ex\@minus.3ex}{\normalfont\large\sffamily\bfseries}}
\def\subsubsection{\@startsection{subsubsection}{3}{\z@}{-2ex\@plus-.5ex\@minus-.2ex}{-2ex\@plus0ex\@minus.3ex}{\normalfont\large\sffamily\bfseries}}
\makeatother
\def\tightitems{\setlength{\parskip}{0pt}}

\def\TODO#1{\textsl{[[TODO: #1]]}}
% \pagestyle{empty}
\setlength{\columnsep}{4em}

% COMMANDS:
%----------
% |              a ShortVerb
% \sl#1          for meta-syntax in code blocks
% \opt#1         for optional parameters
% code           fancyvrb environment allowing \commands
% \columnbreak   for breaking columns
\def\opt#1{\textcolor[gray]{.5}{#1}}
\def\sl#1{{\textrm{\textsl{#1}}}}
\MakeShortVerb{\|}
\renewcommand{\_}[1]{\ifmmode_{_{#1}}\else\underline{\phantom{n}}$\,$#1\fi}

\DefineVerbatimEnvironment{code}{Verbatim}
{gobble=2,baselinestretch=.8,commandchars=\\\{\}}

% hkp://keys.gnupg.net
\def\keyserver{pgp.mit.edu\xspace}
\newcommand{\keyID}[1][key\_id]{\sl{#1}\xspace}

\renewcommand{\thefootnote}{}
\begin{document}

\title[Version 0.85, Nov 2013 --- Dean Serenevy]
{OpenPGP + GnuPG 1.4.10\\
Tutorial and Reference Guide}

% kgpg: +
% gpa: +

% http://www.gnupg.org/gph/en/manual.html#AEN526
% http://wiki.debian.org/subkeys
% http://keyring.debian.org/creating-key.html
% http://ekaia.org/blog/2009/05/10/creating-new-gpgkey/
% https://help.ubuntu.com/community/GnuPrivacyGuardHowto
% http://pgp-tools.alioth.debian.org/
% http://www.apache.org/dev/openpgp.html
% http://www.apache.org/dev/openpgp.html#wot
% http://www.elho.net/crypto/policy/
% http://en.wikipedia.org/wiki/Pretty_Good_Privacy#Web_of_trust


\tableofcontents

\section{Overview}

I have seen many quick ``getting started'' guides that completely skip over
the web of trust management bits of managing a GnuPG database. I have also
seen many high-level ``web of trust is great'' articles that bypass details
of controlling that trust and technicalities of what commands to run. This
tutorial is the result of my research into attempting to bring these two
worlds together into a complete practical gpg tutorial following modern
best-practices.

\textbf{Assumptions:} You are somewhat comfortable using the linux command
line (or can convert the commands I give to appropriate GUI clicks) and you
have at least a vague understanding that gpg ``does something relating to
verifying identities and signing and encrypting documents''.


GnuPG is an implementation of the OpenPGP standard as defined by RFC 4880.
Your OpenPGP keys are really pairs of keys, a \textsl{private key} and a
\textsl{public key}. You use the private key to digitally sign files, and
others use the public key to verify the signature. Or, others use the
public key to encrypt something, and you use the private key to decrypt it.

In practice it is best to use \textsl{subkey} pairs in your regular
``daily'' life. Subkeys are normal keys, except they are bound to (signed
by) a master key pair. By default, GnuPG creates a signing-only key pair as
a master key, and an encryption-only subkey.

\begin{description}
\item[Master key] Identified on ``pub'' line by |gpg --list-keys|. You need
  to save this secret to secure offline media only. Used ONLY to:\\[-4ex]
  \begin{itemize}\tightitems
  \item sign someone else's key
  \item create or revoke subkeys
  \end{itemize}
\item[Encrypting subkey] One is created automatically by gpg as initial
  ``sub'' key; Used for encrypting things to be sent to you; This secret is
  needed anywhere you will need to decrypt.
  % \TODO{How do multiple encryption keys work? Is it useful? Will gpg encrypt for all valid subkeys?}
\item[Signing subkey(s)] Use these keys to sign documents (email, software,
  \dots). You may wish to have many such subkeys for different purposes or
  computers (each independently revocable and each with potentially
  different expirations).
  % \TODO{What makes sense, one per workstation or one per project, or one
  % for each or{\dots} or perhaps 1 signing subkey is sufficient?}
\item[Monkeysphere subkey(s)] Perhaps you are not yet this paranoid, but
  this is another example subkey use.
\end{description}

Each master key has a fingerprint which uniquely identifies the key.
Collisions on this fingerprint represent a breakdown of the OpenPGP system.
You will also see shorter 8 or 16 character ids (taken from the end of the
fingerprint). These are displayed when listing keys and are convenient when
using the gpg command line. However, conflicts may arise in these shorter
ids so they should be avoided when scripting or verifying keys.

The gpg default is to show 8-character IDs, you can get 16 character IDs
using: |gpg --list-keys --keyid-format long| and the full fingerprint using
|gpg --list-keys --fingerprint|.


% Key uses:
% \begin{description}\tightitems
% \item[S] Sign
% \item[C] Certify: (Used in web of trust)(?)
% \item[E] Encrypt
% \item[A] Authenticate: (Used by linux PAM)(?)
% \item[T] (Two-factor authentication)(?): Secret key stored on a smart card
% \end{description}


\section{Web of Trust}

% \TODO{man page say ``calculated validity'', does that really mean validity
%   or does it mean user-chosen trust level?}

I suggest adding |list-options show-uid-validity| to your .gnupg/gpg.conf.
This will cause |gpg --list-keys| to display your validity level for each
key.

\begin{code}
  pub   4096R/E5D0AA3A 2013-11-21
  uid       [ultimate] Dean Serenevy <dean@serenevy.net>
  sub   4096R/15D74B65 2013-11-21
  sub   4096R/6D655C2E 2013-11-21
  sub   4096R/5287FC0C 2013-11-21
  sub   4096R/F49E0D25 2013-11-21

  pub   4096R/06C4AE2A 2010-11-20 [expired: 2011-11-20]
  uid       [ expired] Debian Mozilla team APT archive <pkg-mozilla-maintainers@lists.alioth.debian.org>

  pub   1024R/CE49EC21 2009-01-18
  uid       [ unknown] Launchpad PPA for Mozilla Team
\end{code}

Beware, some sources and interfaces use the term ``valid'' to mean
technical validity of the key data (well-formed, not revoked, not expired,
\dots). In the web of trust sense we require that a valid key be bound to
its User ID (name+email) with an acceptable trust path.

Of course, your own keys are considered valid. Additionally, any keys you
sign will be considered valid. Beyond that, validity can be propegated
further based on the trust you place on keys in your keyring.

\begin{description}\tightitems
\item[Unknown] (default) You don't know how much to trust this user to
  verify other people's keys.

\item[Never] The owner is known to improperly sign other keys.

\item[Marginal] You don't trust this user's ability to verify the validity
  of other people's keys enough to to consider keys valid based on his/her
  sole word. However, provided this key is valid, you will consider a key
  signed by this user valid if it is also signed by at least two other
  marginally trusted users with valid keys.

\item[Full] You trust this user's ability to verify the validity of other
  people's keys so much, that, provided this key is valid, you'll consider
  valid any key signed by him/her.

\item[Ultimate] Used only for your own keys.
\end{description}

These trust levels are completely independent from signing. It can make
sense to leave a key that you have signed as ``Unknown'' (if you sign a key
at a keysigning party, but do not know the person personally).
Additionally, you may choose to alter the trust of keys you have not signed
(for instance, you may choose marginally trust but not sign
\texttt{pub 1024D/449FA3AB 1999-10-05 Linus Torvalds $<$torvalds@transmeta.com$>$})


When you sign a key, you may specify a \textsl{certification level}. This
allows you to indicate what level of verification was performed before
signing the key. For instance, you might sign using ``level 2'' if you
perform a casual verification of a photo ID at a keysigning party. However,
you might sign using ``level 3'' when you do extensive verification of
multiple photo ids and verification of email addresses, or for someone whom
you know personally. You would sign using ``level 1'' when you could not,
or did not verify the key at all. For example, you might sign at level 1
for a ``persona'' verification, where you sign the key of a pseudonymous
user.

Your web of trust can be configured by setting the following values in your gpg.conf:

\begin{description}
\item[completes-needed \textsl{n}] Number of completely trusted users to introduce a
  new key signer (defaults to 1).

\item[marginals-needed \textsl{n}] Number of marginally trusted users to introduce a
  new key signer (defaults to 3)

\item[ask-cert-level] When making a key signature, prompt for a
  certification level. (see gpg manual for more examples of when to use
  which level)

% \item[list-options show-uid-validity] Display your chosen trust level when keys are listed.

\item[max-cert-depth \textsl{n}] Maximum depth of a certification chain (default is
  5).

\item[min-cert-level] When building the trust database, treat any
  signatures with a certification level below this as invalid. Defaults to
  2, which disregards level 1 signatures. Note that certification levels is
  a PGP 5.x standard and that older signatures default to level 0 --- ``no
  particular claim''. For backwards compatibility, these signatures are
  always accepted.
\end{description}


\section{Creating Keys}
\subsection{Creating Your Master Key}
It is recommended to create keys using SHA2 that are at least 2048 bits.
Further, you should keep your master key on a secure USB drive (not network
accessible, used for no other purpose, \dots). Supposing your drive is
mounted at \textsl{/media/thumb},

Create directory with proper permissions: |cd /media/thumb; mkdir gnupg; chmod 700 gnupg|

Optionally, create gnupg/gpg.conf with content in the
``\zref[title]{gpg.conf-suggestion}'' section below.


Execute |gpg --home=/media/thumb/gnupg --gen-key| and choose ``(1) RSA and
RSA (default)'' and set key size to 4096. Use a long and strong passphrase.
Once finished, gpg will display something like:

\begin{code}
  gpg: key \sl{E5D0AA3A} marked as ultimately trusted
  public and secret key created and signed.

  gpg: checking the trustdb
  gpg: 3 marginal(s) needed, 1 complete(s) needed, PGP trust model
  gpg: depth: 0  valid:   1  signed:   0  trust: 0-, 0q, 0n, 0m, 0f, 1u
  pub   4096R/E5D0AA3A 2013-11-21
        Key fingerprint = 3A4D 51DF 68FC D20B FC0C  1C5F E4E6 033F \sl{E5D0 AA3A}
  uid       [ultimate] Dean Serenevy <dean@serenevy.net>
  sub   4096R/15D74B65 2013-11-21
\end{code}


\subsection{Add a subkey}
See also: https://wiki.debian.org/subkeys
\begin{code}
  $ gpg --home=/media/thumb/gnupg --edit-key \sl{E5D0AA3A}
  \dots
  gpg> addkey
  {\dots} \sl{RSA (sign only), 4096 bits}, \dots
  gpg> save
\end{code}
%$% stupid emacs
You can also use this to add an encryption-only subkey, though I don't see
much use for more than the automatically created one. After creating some
subkeys, your key will now look different (Note: you will need to
re-publish your key any time you create or revoke a subkey).

\begin{code}
  $ gpg --list-keys --fingerprint \sl{E5D0AA3A}
  pub   4096R/E5D0AA3A 2013-11-21
        Key fingerprint = 3A4D 51DF 68FC D20B FC0C  1C5F E4E6 033F E5D0 AA3A
  uid       [ultimate] Dean Serenevy <dean@serenevy.net>
  sub   4096R/15D74B65 2013-11-21
  sub   4096R/6D655C2E 2013-11-21
  sub   4096R/5287FC0C 2013-11-21
  sub   4096R/F49E0D25 2013-11-21
\end{code}
%$% stupid emacs

Unfortunately, I could see no way to add useful labels to each subkey to
explain its purpose --- you will need to keep separate document listing
what each subkey is meant to be used for (for instance, I am keeping
|6D655C2E| on my laptop to sign email messages, |5287FC0C| on my home
desktop to sign debian packages published on my server, and |F49E0D25| on
my work machine to sign packages for company-internal distribution).

We only want to store the minimal set of subkey secrets required on each
relevant system (our master secret should remain offline). Thus, we need to
export individual secrets from our thumb drive to be imported to each
machine.

\begin{code}
  \# NOTE: The exclamation mark ! is significant)
  gpg --home=/media/thumb/gnupg --export-secret-subkeys \keyID[subkey\_id]! > \sl{subkey1}\_secret.gpg
  gpg --home=/media/thumb/gnupg --export \keyID[master\_key\_id] > pubkeys.gpg

  \# Now import these into your ~/.gnupg directory
  gpg --import pubkeys.gpg subkey1\_secret.gpg
\end{code}

Verify that |gpg -K| shows a |sec\#| instead of just |sec| for your private
key. That means the secret key is not really there.

Also, at this point you probably should change the passphrase used on the
exported keys (see ``\zref[title]{passwd}'' below).


\subsection{Add an additional name or email address to your key}
\begin{code}
  $ gpg --home=/media/thumb/gnupg --edit-key E5D0AA3A
  \dots
  gpg> adduid
  \dots
  gpg> save
\end{code}
%$% stupid emacs
When you add a new uid, it will become the primary uid for your key. To
edit the primary uid, edit your key again, select the uid you want
|gpg> uid 2|, mark it primary |gpg> primary|, and finally |gpg> save|.

\subsection{Send new key to key server}
\begin{code}
  gpg --keyserver \keyserver --send-key \sl{NEW\_ID}
\end{code}

You may omit the keyserver option if you specify a keyserver in gpg.conf.
Keyservers synchronize with each other periodically so uploading to a
single keyserver is sufficient --- your key will propagate to the others.

\subsection{Revoking a key}
The process is generate and apply your revocation, then re-publish your modified (revoked) key:
\begin{code}
  gpg --output revokedkey.asc --gen-revoke \keyID
  gpg --import revokedkey.asc
  gpg --keyserver \keyserver --send-key \keyID
\end{code}

\subsection{Changing Passphrase}\zlabel{passwd}
\begin{code}
  gpg --edit-key \keyID
  gpg> passwd
  \dots
  gpg> save
\end{code}

\section{Key-Signing}
If you go to a keysigning party, you should bring slips of paper or business cards with:

\begin{quote}
  Your Name\\
  Your GPG Fingerprint\\
  Your Email Address
\end{quote}

You should \textsl{NOT} bring your thumb drive containing your private
secrets! While at the party, trade fingerprints with others. Perform any
verification necessary (look at IDs, \dots) and note the level of
verification performed. Then, after you get home, mount your thumb drive
containing your secret key and sign any keys that you feel comfortable
signing. This approach protects your secret key and maintains your
signature integrity by avoiding peer-pressure signatures.

Note: The |signing-party| Debian package provides some tools to help you
with this process. |gpg-key2ps| turns a GnuPG key into a PostScript file to
print paper slips with your fingerprint, and |gpg-mailkeys| or |caff| will
email signed keys to their owners.

\subsection{Signing someone else's key}\zlabel{sign}
\begin{code}
  gpg --home=/media/thumb/gnupg --keyserver \keyserver --recv-keys \sl{their\_id}
  gpg --home=/media/thumb/gnupg --list-keys --fingerprint \sl{their\_id}
  \# Verify the full fingerprint of their key before proceeding!

  gpg --home=/media/thumb/gnupg --sign-key \sl{their\_id}a
  gpg --home=/media/thumb/gnupg --list-sigs \sl{their\_id}
  \# Verify your signature appears on the list of signatures

  gpg --home=/media/thumb/gnupg --export --armor \sl{their\_id} > signed\_key.gpg
\end{code}

Email the signed key to its owner so that they can publish your signature.

\subsection{Sign my new key with my old key}
If you generate a new key, you should sign the new key with your old key.
If you have both present in the same keyring, you can select which to sign
with via the |-u| option:
\begin{code}
  gpg --home=/media/thumb/gnupg -u \sl{old\_id} --sign-key \sl{new\_id}
\end{code}

\subsection{Someone signed my key}
When someone send you a signature on your key, you need to import their
signature and then re-upload your key to a public keyserver.
\begin{code}
  gpg --home=/media/thumb/gnupg --import signed\_key.gpg
  gpg --keyserver \keyserver --send-key \keyID
\end{code}




\section{Using Keys}
\begin{tabbing}
List all public keys I have: \= \texttt{gpg --list-keys}\\
Search public keys I have:   \> \texttt{gpg --list-keys \sl{search\_term}}
\end{tabbing}

Generally search terms can be a key id, a name, or an email address. Due to
collision of names, generally a key id or email address is preferred.


\subsection{Importing Keys}
Import a key: \texttt{gpg --import user-key.gpg}

Get public key from a public key server (beware collision! --- always
verify the fingerprint):
\begin{code}
  gpg --recv-keys --keyserver \keyserver user@some-email.com
  gpg --recv-keys --keyserver \keyserver 39a629558da0af10
\end{code}

Verify the fingerprint of the received key:
\begin{code}
  gpg --fingerprint user@some-email.com
\end{code}

\subsection{Verifying Signatures}
\begin{code}
  gpg --verify signedfile
  gpg --verify signedfile.sig signedfile
\end{code}

\subsection{Signing Documents}

Sign a message: \texttt{gpg --clearsign \sl{filename}}\\
Sign a message in separarte file: \texttt{gpg --detach-sign --armor \sl{filename}}


\subsection{Encrypting}
Encrypt using a public key (and sign it):
\begin{code}
  gpg --recipient \sl{their\_username} --sign --encrypt \sl{filename}
\end{code}

Encrypt using a passphrase: \texttt{gpg --symmetric \sl{filename}}

\subsection{Decrypting}
Decrypt a file: \texttt{gpg --decrypt filename}\\
Decrypt STDIN:  \texttt{gpg --decrypt}


\section{Suggested gpg.conf}
\zlabel{gpg.conf-suggestion}
\begin{code}
  use-agent
  no-greeting
  require-secmem
  ask-cert-level

  # If you have more than 1 secret key in your keyring, you may want
  # to uncomment the following option and set your preffered keyid
  # default-key  748BEE343A8B34EE6FB05F741DCFAA536B861770

  ## These are the defaults, but you might want to change them
  # completes-needed 1
  # marginals-needed 3
  # max-cert-depth   5

  display-charset utf-8
  keyserver \keyserver

  list-options show-uid-validity

  personal-digest-preferences SHA512
  cert-digest-algo SHA512
  default-preference-list SHA512 SHA384 SHA256 SHA224 AES256 AES192 AES CAST5 ZLIB BZIP2 ZIP Uncompressed
\end{code}

\section{Sources}
\begin{description}\tightitems
\item[Creating a new GPG key]                      \url{http://keyring.debian.org/creating-key.html}
\item[Using OpenPGP subkeys in Debian development] \url{http://wiki.debian.org/subkeys}
\item[PGP Tools]                                   \url{http://pgp-tools.alioth.debian.org/}
\item[OpenPGP Trust Models]                        \url{http://web.monkeysphere.info/doc/trust-models/}
\item[RFC 4880]                                    \url{http://www.ietf.org/rfc/rfc4880.txt}
\end{description}

\section{See also}
\begin{description}\tightitems
\item[GPG Frontends] \url{http://www.gnupg.org/related\_software/frontends.html}
\end{description}

\end{document}
