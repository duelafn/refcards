\documentclass{article}
\usepackage{fontspec}
\usepackage[centering,width=10in,height=7.5in]{geometry}
\usepackage{parskip}
\parskip 0pt
\geometry{landscape}
\usepackage{amsfonts,amsmath,amssymb,graphicx}
\usepackage{multicol,xcolor,shortvrb,fancyvrb}
\setlength{\columnsep}{4em}
\let\epsilon\varepsilon\let\phi\varphi
\usepackage{syntax}% http://www.ctan.org/pkg/syntax-mdw
\usepackage{xparse}% http://www.ctan.org/pkg/xparse
\usepackage{newunicodechar}%% Also need \usepackage[utf8]{inputenc} if not using xelatex
\usepackage{tcolorbox}
\usepackage{tabularx}
\usepackage{xspace}
\usepackage[breaklinks,hidelinks]{hyperref}

\newif\ifcolor
\colortrue

% COMMANDS:
%----------
% \rv{Int}         return value
% \comment{# ...}  comment
% \text#1          typeset in text style
% \subtitle#1      subtitle in block
%
% _  |             underscore, verticalbar
% \string_         underscore in file name or reference
% "#1"             a ShortVerb  (two versions available, see below)
% \lr{foo}         left align code, linebreak
% \lr{foo}{bar}    left align code, right align text, linebreak
% \lr*{foo}{bar}   left align code, linebreak, right align text
% <#1>             for meta-syntax  (method 1)
% «#1»             for meta-syntax  (method 2)
% \sl#1            slanted text ("fixes" plain TeX \sl command)
% \opt#1           for optional parameters
% code             fancyvrb environment allowing \commands and «...»
% \Code{...}       like \lit*{...}
% \columnbreak     for breaking columns
% \"\^\<\>\~       literal ", ^, <, >, ~
% \bs              \textbackslash
\makeatletter
\syn@shorts\relax\relax
\addspecial\|\catcode`\|\active
\addspecial\"\catcode`\"\active%% can use commands in "..."
\let\Myulitleft\ulitleft
\renewcommand{\ulitleft}{\color{ttcolor}\Myulitleft}
% \shortverb{"}%% may not use commands in "..."
\def\opt#1{\textcolor[gray]{.5}{#1}}
\def\sl#1{{\textrm{\textsl{#1}}}}
\let\bs\textbackslash

% Colors: may redefine in .tex file after importing refcard
\definecolor{secfrcolor}{rgb}{0.2,0.2,0.4}            % Section Frame Color
\definecolor{secbgcolor}{rgb}{0.2,0.4,0.6}            % Section Background Color
\definecolor{subsecfrcolor}{rgb}{0.4,0.6,0.6}         % Subsection Frame Color
\definecolor{subsecbgcolor}{rgb}{0.6,0.8,0.8}         % Subsection Background Color
\definecolor{backcolor}{rgb}{0.937,0.969,1.000}       % Column Environment Color
\definecolor{commentcolor}{rgb}{0.8353,0.0000,0.0000} % Comment Color
\definecolor{metacolor}{rgb}{0.4588,0.4588,0.4588}    % Metasyntax Color (grey)
\definecolor{textcolor}{rgb}{0.1020,0.1373,0.4941}    % Non-Literal Text Color
\definecolor{ttcolor}{rgb}{0,0,0}                     % Typewriter (Literal) Text Color

\NewDocumentCommand\lr{smG{}}{%
\begin{flushright}
\mbox{}\texttt{\spaceskip.35em\@plus.2em\@minus.15em\relax\color{ttcolor} #2}%
\hfill\IfBooleanTF{#1}{\mbox{}\\\mbox{}\hfill}{} #3\mbox{}
\end{flushright}}

\def\sup#1{\ensuremath{^{\text{\small\textrm{#1}}}}}

\def\"{\char`"}
\def\^{\char`^}
\def\<{\char`<}
\def\>{\char`>}
\def\~{\char`~}
\def\dots{{\rmfamily\ifmmode \mathellipsis \else \textellipsis \fi}}
\newcommand{\Code}[1]{\texttt{\spaceskip.35em\@plus.2em\@minus.15em\relax\color{ttcolor} #1}}
\makeatother

\newunicodechar{«}{\metavar}%» stupid emacs
\def\metavar#1»{{\normalfont\itshape\color{metacolor}#1\/}}

\catcode`\<\active\catcode`\>\active
\def<{\bgroup\normalfont\itshape\color{metacolor}}
\def>{\/\egroup}

\DefineVerbatimEnvironment{code}{Verbatim}
{gobble=2,baselinestretch=.8,commandchars=\\\{\},formatcom=\color{ttcolor}}

\def\section#1{\par\vspace{3ex}{\large{\textbf{#1}}}\par}
\def\subsection#1{\par\vspace{2ex}\textbf{» #1}\par}
\def\br{\vspace{1.75ex}}

\newfontfamily\unicode{DejaVu Sans}% Font with enough unicode symbols
\setmainfont[Mapping=tex-text]{Linux Libertine O}
\setmonofont[Scale=MatchLowercase]{Inconsolata}% fonts-inconsolata. Note: "Mapping=tex-text" makes -- become endash (bad for code)
% \setmonofont[Scale=MatchLowercase]{DejaVu Sans Mono}% Note: "Mapping=tex-text" makes -- become endash (bad for code)

\newenvironment{titlesection}{\begin{center}}{\end{center}}
\NewDocumentEnvironment{block}{o}{\IfValueTF{#1}{\subsection{#1}}{}}{}

\ifcolor
\def\section#1{\par
  \begin{tcolorbox}[coltext=white,colframe=secfrcolor,colback=secbgcolor,boxrule=0.5pt,left=6pt,right=6pt,top=4pt,bottom=3pt,arc=0pt,boxsep=0pt,width=\hsize]
    \large \textbf{#1}
  \end{tcolorbox}\nopagebreak\par}
\def\subsection#1{\par
  \begin{tcolorbox}[coltext=textcolor,colframe=subsecfrcolor,colback=subsecbgcolor,boxrule=0.5pt,left=6pt,right=6pt,top=1pt,bottom=1pt,arc=0pt,boxsep=0pt,width=\hsize]
    \textbf{#1}
  \end{tcolorbox}\par}

\renewenvironment{titlesection}
{\begin{tcolorbox}[coltext=textcolor,colframe=black,colback=black,boxrule=0pt,left=6pt,arc=0pt,boxsep=0pt,width=\hsize]\color{white}\centering}
{\end{tcolorbox}\par}

\RenewDocumentEnvironment{block}{o}%
{\IfValueTF{#1}%
{\begin{tcolorbox}[coltext=textcolor,colframe=backcolor,colback=backcolor,boxrule=0pt,left=3pt,right=3pt,top=3pt,bottom=3pt,arc=0pt,boxsep=0pt,width=\hsize,toptitle=2pt,bottomtitle=2pt,coltitle=textcolor,colbacktitle=subsecbgcolor,title={#1}]}%
{\begin{tcolorbox}[coltext=textcolor,colframe=backcolor,colback=backcolor,boxrule=0pt,left=3pt,right=3pt,top=3pt,bottom=3pt,arc=0pt,boxsep=0pt,width=\hsize,toptitle=2pt,bottomtitle=2pt,coltitle=textcolor,colbacktitle=subsecbgcolor]}%
}
{\end{tcolorbox}}
\fi

\def\rv#1{{\color{metacolor}$\to$#1}}
\def\comment#1{{\ifcolor\color{commentcolor}\fi#1}}
\def\text#1{{\normalfont\color{textcolor}#1}}
\let\subtitle\tcbsubtitle

\pagestyle{empty}
% \raggedcolumns
\begin{document}\small
\ifcolor\color{textcolor}\fi
\begin{multicols*}3

\begin{titlesection}
  {\large\bfseries Perl 6.c Reference Guide}\\
  Version 0.7.1, Dec 2016 --- Dean Serenevy\\
  \texttt{CC0} / Dedicated to Public Domain
\end{titlesection}

% http://blogs.perl.org/users/zoffix_znet/2016/02/perl-6-shortcuts-part-1.html

\section{Rakudo Runtime}
\begin{block}
\lr{perl6 --profile --profile-filename=f.html src.pl}
\end{block}
\begin{block}[Debugger]
% https://perl6advent.wordpress.com/2012/12/05/a-perl-6-debugger/
\lr{perl6-debug-m src.pl}
\lr{[ENTER], s}{step into, step over}
\lr{so}{step out of current sub}
\lr{r, rt}{run to exception, even if handled}
\lr{bp add <Foo:46>}{add breakpoint at Foo.pm line 46}
\end{block}
% https://github.com/jnthn/grammar-debugger

\section{Basic Expressions}
\begin{block}[Variables]
\Code{\$scalar, @array, \%hash, \&code}\\
\Code{\$var1, \$var_one, \$var-one, \$var1_, \$_var}
\lr{@array[], @array[<0>], @array[<2>,<3>]}{all, item, slice}
\lr{\%hash\{\}, \%hash\{'<x>'\}, \%hash\{'<x>','<y>'\}}{all, item, slice}
\lr{\%hash\<<x>\>, \%hash\<<x y>\>}{item, slice}
\lr{\%hash\<<x>\>:exists, \%hash\<<y>\>:delete}
\lr{{\~} <\$x>, + <\$x>, ? <\$x>}{stringy, numeric, boolean coersion}
\lr{| @x, |(0..5)}{list expansion}
\end{block}

\begin{block}[Numbers / Constants / Special Values]
\lr{10_000, 0xBEEF, 0o755; -2.5, 1/3}{Int; Rat}
\lr{1e0, 6.022e23; 1+2i, 6.123e5i}{Num; Complex}
\lr{constant <answer> = <42>;}{defining constants}
\lr{Nil, True, False}{generic undefined value, true, false}
\lr{pi, $\pi$; tau, $\tau$; e}{3.14159\dots; 6.28318\dots; 2.71828\dots}
\lr{\$*OUT, \$*ERR}
\end{block}

\begin{block}[Pairs]
\lr{<foo> {=\>} <42>, '<foo>' {=\>} <32>, (<STRING_EXPRESSION>) {=\>} <32>}
\lr{:<foo>, :!<foo>}{"<foo> {=\>} True|False"}
\lr{:<foo>(<\$bar>)}{"<foo> {=\>} <\$bar>"}
\lr{:<foo>\<<quoted list>\>}{"<foo> {=\>} \<<quoted list>\>"}
\lr{:<foo>\string«<interpolated list>\string»}{"<foo> {=\>} \string«<interpolated list>\string»"}
\lr{:<foo>['<some>', '<vals>']}{"<foo> {=\>} ['<some>', '<vals>']"}
\lr{:<foo>\{<key> {=\>} '<val>'\}}{"<foo> {=\>} \{<key> {=\>} '<val>'\}"}
\lr{:\$<foo>}{"<foo> {=\>} \$<foo>"}
\end{block}

\begin{block}[Lists and Iteration (Synopsis 7)]
\lr{Seq; List; Array}{read once; immutable; mutable}
\lr{\$*IN.lines; (1, 2, 3); [1, 2, 3]}{"Seq", "List", "Array"}
\lr{@fib := (1, 1, * + * ... *).list}{lazy list (memoized)}
\lr{for @a \{\}; for |@a, |@b \{\}}{descend into arrays}
\lr{for \$@a \{\}; for @a, @b \{\}}{loop over arrays themselves}
\lr{@a.push: 1,2; @a.push: [1,2]; @a.push: @b}{descend}
\lr{@a.push: \$[1,2]; @a.push: \$@b}{push Array}
\lr{push, pop, shift, unshift, splice}{Array methods}
\lr{@a[5] = 42; @a.elems; @a.end}{autoviv., num, last idx}
\lr{for 1, 2 Z 'a', 'b' {-\>} \$x \{ say \"\$x[0]: \$x[1]\"{} \}}
\lr{for 1, 2 Z 'a', 'b' {-\>} *(\$x,\$y) \{ say \"\$x: \$y\"{} \}}
\lr{for flat 1, 2 Z 'a', 'b' {-\>} \$x,\$y \{ say \"\$x: \$y\"{} \}}
\lr{my @fixed[9]; my @shaped[3;3]}{fixed sized}
\lr{@shaped Z= @fixed}{reshape an array}
\end{block}
\begin{block}
\begin{code}
  my @values = lazy gather while «COND» \{ take <VAL> \}
  my @values = lazy gather for «LIST» \{ take <VAL> \}

  my @all-nodes = lazy gather traverse(\$tree);
  sub traverse(Tree \$t) \{
    traverse(\$t.left) if \$t.left;
    take transform(\$t);
    traverse(\$t.right) if \$t.right;
  \}
\end{code}
\end{block}


\begin{block}[Sequences]
\url{https://perl6.party/post/Perl-6-Seqs-Drugs-and-Rock-n-Roll--Part-2}
\lr{\$seq := 0 ... *; \$seq := 1, 1, \{ \$\^a + \$\^b \} ... *}
\lr{\$seq := (1, 2, 3).map(* + 1)}{lazy, read-once}
\lr{@a := (1, 2, 3).map(* + 1)}{eager, read many}
\lr{.say for |\$seq; .say for |\$seq}{error 2\sup{nd} time}
\lr{.say for \$seq.cache; .say for \$seq.cache}{ok}
\end{block}

\begin{block}[Quoting]
\lr{\comment{\# comment}, \comment{\#`(inline comment)}}
\lr{Q[ ], Q{\<} \>, Q\{ \}, ...}{literal -- no escaping at all}
\lr{Q:q[ \bs] ], q[ \bs] ], ' \bs' '}{terminator escaping}
\lr{Q:qq[ ], qq[ <2\texttt{*}>\$<x> <=> \{ 2*\$<x> \}, @<arr>[] ]}{interpolation}
\lr{\"\$<x>.lc(), @<arr>.join(\", \") <@foo.com>\"}{interpolation}
\lr{Q:w[ ], qw[ ]}{word quoting (strings)}
\lr{Q:w:v[ ], {\<} <x y z> {\>}}{value quoting (smart via "val()")}
\lr{Q:qq:w[ ], qqw[ ], \string« \$<x> <y z> \string»}{interp.~word/val quoting}
\lr{Q:x[ ], qx[ <ps auxx> ]}{shell command}
\lr{Q:qq:x[ ], qqx[ <ps | grep '>\$<prog'> ]}{interp.~shell cmd}
\lr{Q:to/<END>/ <...> <END>}{heredoc (may indent)}
\lr{\bs c[<CHAR NAME>]}{Example: \bs c[SNOWMAN] $\to$ {\unicode ☃}}
\end{block}

\section{Flow Control}
\begin{block}
\lr{if <EXPR> \{ \} elsif <EXPR> \{ \} else \{ \}}{boolean test}
\lr{with <EXPR> \{ \} orwith <EXPR> \{ \} else \{ \}}{definedness}
\lr{unless <EXPR> \{ \}, without <EXPR> \{ \}}{no else allowed}
\lr{for <LIST> \{ \}; for <LIST> \{ \$\^<x> \}}{implicit "\$_" or "\$\^<x>" arg}
\lr{for <LIST> {-\>} \$<x>, \$<y> \{ \}}{explicit args (n at a time ok)}
\lr{for <LIST> {-\>} \$<x> is rw \{ \}}{mutator (also "is copy")}
\lr{while|until <EXPR> \{ \}}
\lr{repeat while|until <EXPR> \{ \}}{do at least once}
\lr{loop \{ \}; loop (<INIT>;<TEST>;<ITER>) \{ \}}{parens required}
\lr{next, last, redo}{for/while/until/loop controls}
\lr*{given <EXPR> \{ when <EXPR> \{ \} default \{ \} \}}{use "proceed" for fall-through, "succeed" for break}
\lr{<EXPR> if|unless|for|while|until <EXPR>;}
\end{block}
\begin{block}[Exceptions]
\begin{code}
  try \{\comment{# \"try\" optional, can CATCH in any block}
   ...
   \comment{# CATCH may appear anywhere within block}
   \comment{# is in same scope so can use block vars}
   CATCH \{
     when X::AdHoc \{ \comment{\#`(Type thrown for die «msg»)} \}
     when X::IO \{ ... \}
     default \{ ... \}
   \}
   \comment{# LEAVE is a general phaser, not CATCH specific}
   LEAVE \{ \comment{\#`(called when leave by any means)} \}
   ...
   die \"Oops\";\comment{# X::AdHoc.new(:payload<Oops>).throw}
   ...
  \}
\end{code}
\end{block}

\section{Phasers}
\begin{block}
\lr{FIRST NEXT LAST}{Loop: before first, before next, after last}
\lr{BEGIN CHECK INIT END}{compile early/late; run early/late}
\lr{ENTER LEAVE KEEP UNDO}{Block: in/out; success/fail}
\lr{PRE POST CATCH}{precondition; finally; exceptions}
\lr{LINK COMPOSE CONTROL QUIT CLOSE}
\end{block}

\section{Operators (Round 1)}
\begin{block}
\lr{=, :=, ::=}{assignment, binding, read-only bind}
\lr{cmp}{automatic string-or-numeric}
\lr{\~\~, !\~\~}{smart or regex match, non-match}
\lr{leq, lt, le, eq, ge, gt, ne}{string comparison}
\lr{\~, x}{string concat, string replication}
\lr{\<=\>, ~\<, \<=, ==, \>=, ~\>, !=}{numerical comparison}
\lr{+, -, *, /, \%, **}{math operators}
\lr{\$<a> mod \$<b>, \$<a> lcm \$<b>, \$<a> gcd \$<b>, \$a++, \$a--}{}
\lr{\$<a> div \$<b>, \$<a> \%\% \$<b>}{integer division, bool \Code{0 == \$<a> \% \$<b>}}
\lr{<COND> ?? <IFTRUE> !! <IFFALSE>}{ternary operator}
\lr{+\&, +|, +\^, +\<, +\>}{bitwise ops, bit shift}
\lr{\&, |, \^}{Junction: "all", "any", "one" (no operator for "none")}
\lr{1..5, 1..\^6, 0\^..\^6, \^6}{1,2,3,4,5 except "\^6" includes 0}
\end{block}

\section{I/O}
\begin{block}
\lr{put \$msg; print \$msg}{write ".Str" with/out newline}
\lr{say \$msg; note \$msg}{write ".gist" w/nl to \$*OUT/\$*ERR}
\lr{prompt(\"? \"); get()}{chomped input with/out prompt}
\lr{slurp \$file; spurt \$file, \$content}{easy read/write}
\end{block}
\begin{block}[Files]
\lr{my \$fh := open(\"<path/to/file>\", <:r>)}{Open file}
\lr{:r, :w, :rw, :a}{read, write, read-write, append}
\lr{:bin, :chomp, :nl\<<\bs r\bs n>\>, :enc\<<UTF-8>\>}{other opts}
\subtitle{\texttt{.IO} is your friend}
\lr{\"<myfile>\".IO.lines(:enc\<UTF-8\>)}{lines (chomped!)}
\lr{\"<myfile>\".IO.slurp(:enc\<UTF-8\>)}{slurp!}
\lr{\"<myfile>\".IO.open(:enc\<UTF-8\>).print(<stuff>)}{spew!}
\lr{\"<myfile>\".IO.e}{Exists (also ".d .f .l .r .w .x .s .z")}
\lr{dir(\"</etc>\")}{list of IO objects, skips "." and ".."}
\lr{dir(\"</etc>\",test={\>} *.IO.d)}{list dirs only (includes ".", "..")}
See also: IO::Path
\end{block}

\section{Subroutines}
\begin{block}
\lr{sub «s» \{...\}, sub «s» \{???\}, sub «s» \{!!!\}}{stub/predelcare}
\subtitle{Type Signatures}\vspace{-1.5ex}
\begin{code}
  \comment{# auto-numifies: nice("2", "3") \(\to\) ok}
  sub «nice»($«x», $«y») returns Num \{ $«x» + $«y» \}
  \comment{# caller must numify: naughty("2", "3") \(\to\) error}
  sub «naughty»(Num $«x», Num $«y») \{ $«x» + $«y» \}
  \comment{# with coersion: nice("2", "3") \(\to\) ok}
  sub «nice»(Num() $«x», Num() $«y» {-}{-}\> Num) \{ $«x» + $«y» \}
\end{code}
\end{block}
\begin{block}[Optional and default arguments]
\begin{code}
  sub «say-hello»(\$«name»?) \{
    with \$«name» \{ say "Hello " ~ \$«name» \}
    else        \!\{ say "Hello Human" \}
  \}
  sub «say-hello»(\$name = 'Human') \{ ... \}
\end{code}
\end{block}
\begin{block}[Anonymous / Lambda Functions]
\begin{code}
  $«s»=  -> $a, $b \{ $a + $b \}; \comment{# pointy sub}
  $«s»= <-> $a, $b \{ ... \};     \comment{# mark \$a, \$b "is rw"}
  $«s»= \{ $^a + $^b \};          \comment{# implicit parameters}
  $«s»= * + *;                  \comment{# whatever code}
  say $«s»(3, 4);
\end{code}
%$% Stupid emacs
\end{block}
\begin{block}[multiple dispatch, "proto" declaration, memoization]
\begin{code}
  \comment{# Need "proto" declaration to cache a multi sub}
  \comment{# Coerce to Int to reduce cache space}
  proto «fib»(Int() \$) is cached \{ * \}
  multi «fib»(0) \{ 1 \}  multi «fib»(1) \{ 1 \}
  multi «fib»(\$«a» where \{ \$«a» ~~ 0|1 \}) \{ 1 \}\comment{# slow alt}
  multi «fib»(\$«a») \{ «fib»(\$«a»-1) + «fib»(\$«a»-2) \}

  \comment{# \{*\} does the dispatch (dangerous, but cool)}
  proto «foo»(|) \{ say \{*\}.elems \}
\end{code}
\end{block}
\begin{block}[traits]
\lr{is cached}{memoization, even on array contents}
\lr{is pure}{no random or side-effect; for optimizations}
\lr{is rw}{lvalue subs, use "return-rw" to return var}
\lr{is export, is export(:<foo> :<bar>)}{flag as exportable. Special tags ":ALL", ":DEFAULT", ":MANDATORY"}
\lr{is DEPRECATED(\$<msg>)}{friendly deprecation warnings}
\lr{is hidden-from-backtrace}
\end{block}
\begin{block}[Routine Methods]
\lr{\&<foo>.arity, \&<foo>.count}{min/max positional args}
\lr{\&<foo>.Str, \&<foo>.signature}{name and signature}
\lr{\&<foo>.file, \&<foo>.line}{where declared}
\lr{\&<foo>.assuming(\dots)}{currying}
\lr{\&<foo>.multi, \&<foo>.yada}{is multi, is stub (has ``\dots'' body)}
\begin{code}
  state $x = $_[0] // 1;
  callsame, callwith, nextsame and nextwith
\end{code}
\end{block}

\section{Pattern Matching (RegExp)}
\begin{block}
\lr{/.../, rx/.../, rx[...]}{Regex object constructors}
\lr{my regex <foo> \{ <PAT> \}}{named Regex}
\lr{my token <foo> \{ <PAT> \}}{implies \Code{:r} (no backtracking)}
\lr{my rule <foo> \{ <PAT> \}}{implies \Code{:r:s} (+significant space)}
\lr{\$<x> {\~\~} /.../, \$<x> {\~\~} m\%...\%, \$<x>.match(\$<pat>)}{do match}
\lr{\$<x> {\~\~} s/.../.../; S///}{replacement; non-mutating}
\end{block}

\begin{block}[Regex Patterns]
\lr{/ '<a literal>' /}{quotes optional if pattern is alnum}
\lr{\bs d, \bs D, \bs w, \bs W}{unicode digit/non, wordy-character/non}
\lr{\bs n, \bs N, \bs s, \bs S}{logical newline/non, whitespace/non}
\lr{\bs h, \bs H, \bs v, \bs V}{horizontal space/non, vertical/non}
\lr{. ~ \<[ a..f 0..9 ]{\>} ~ \<-[{\"}']\>}{any char, char class, neg}
\lr{\<:<Sm>\>, \<:!<Sm>\>, \<:<M>|:<P>-:<Sm>\>}{unicode properties}
\end{block}

\begin{block}[Quantifiers]
\lr{<x>+ ~ <x>* ~ <x>?}{1 or more, 0 or more, 0 or 1}
\lr{<x> ** <2>..<5>}{between 2 and 5}
\lr{<x>+ {\%} '<,>' ~ <x>+ {\%\%} '<,>'}{w/separator, allow trailing sep}
\end{block}

\begin{block}[Anchors]
\lr{{\^} ~ {\^\^} ~ {\$\$} ~ {\$}}{start str, start line, end line, end str}
\lr{{\<\<} ~ {\string«} ~ {\>\>} ~ {\string»}}{2 ways beginning of word, end of word}
\lr{{\<?after <PAT>\>} {\<?!after <PAT>\>}}{at/not at end of <PAT>}
\end{block}

\begin{block}[Grouping]
\lr{(...) ~ [...]}{capturing (\$0,\$1,\dots), non-capturing group}
\lr{[<P> || <Q>] ~ [<P> | <Q>]}{first alternative, longest alternative}
\lr{{\$\<<nam>\>} = <PAT>}{named capture \Code{\~\$\<nam\>}}
\lr{\<<foo>\>}{pre-declared named regex capture \Code{\~\$\<foo\>}}
\lr{\<<bar>=<foo>\>}{aliased named regex capture \Code{\~\$\<bar\>}}
\lr{\<ident\>\ \<upper\>\ \<lower\>}{identifier, char by case}
\lr{\<alpha\>\ \<:alpha\>\ \<digit\>\ \<xdigit\>}{\bs w, unicode, \bs d, hex}
\lr{\<print\>\ \<graph\>\ \<cntrl\>\ \<punct\>\ \<alnum\>}
\lr{\<wb\>\ \<ww\>\ \<ws\>}{0-width: wrd bndry, wrd wrd, wrd sep}
\lr{\<space\>\ \<blank\>\ \<before ...\>\ \<after ...\>}{\bs s, space/tab}
\end{block}

\begin{block}[Regex Adverbs]
\lr{rx:i/<a>/, /:i <a>/, /[:s<a> <b>] :i <c>/}{before, inside, local}
\lr{:ignorecase, :i}{case insensitive}
\lr{:ratchet, :r}{no backtracking}
\lr{:sigspace, :s}{match \Code{\<.ws\>} if space \textsl{follows} a chunk}
\end{block}

\begin{block}[Matching Adverbs]
\lr{\$<x> {\~\~} m:g/<a>/, \$<x>.match(/.../, :g)}{external only!}
\lr{:global, :g}{find all (non-overlapping) matches}
\lr{:overlap, :ov}{find one match at each possible start}
\lr{:exhaustive, :ex}{find all (overlapping) matches}
\lr{}{Note: ":g" $\subseteq$ ":ov" $\subseteq$ ":ex"}
\lr{:continue(<n>), :c(<n>)}{start search at char <n>}
\lr{:pos(<n>), :p(<n>)}{anchor search at char <n>}
\end{block}

\begin{block}[Examples]
\lr{say '\"in quotes\"' {\~\~} / '\"' \<-[ {\"} ]{\>} * '\"'/;}
\end{block}

\section{Parallelism}
\begin{block}
\lr{Iterable.race; .hyper}{parallelism; preserve order}
\lr{@rv := [1..100].hyper.map: \{ \$\^x**2; \}}
\lr{}{Note: default batch is 64, so only parallel for big data}
\lr{(1,3,2,4).hyper(:batch(1)).map:\{sleep\$\^x; \$\^x**2\}}
\begin{code}
  \$prmis := start \{ sleep 1; say 2; sleep 2; say 4 \}
  say 'One'; sleep 2; say 'Three';
  await \$prmis;

  my @files := qx\{ find . -name '*.md' \}.lines;
  await do for @files -> \$file \{
      start \{
          my \$t = qqx\{file \$file\};
          say "UTF-8" if \$t ~~ / 'UTF-8' | ASCII /;
          say "BOM"   if \$t ~~ / BOM /;
      \}
  \}

  my \$chan := Channel.new;
  \$is_closed = \$chan.closed;
  \$is_closed.then( { note "It's closed!" } );
  \$chan.send(\$_) for @files;

  my @prmis;
  react \{# 1 thread at a time like coroutines
    whenever \$chan -\> \$file \{
      push @prmis, start \{ ... \}
  \}\}
  await Promise.allof(@prmis);

  # True parallelism (tap runs on multiple threads):
  my \$supplier := Supplier.new;
  my \$supply   := \$supplier.Supply;
  say "Is Supply live? \{\$supply.live\}";
  \$supply.tap( -> \$v \{ say \$v \});
  \$supplier.emit(\$_) for 1 .. 10;
\end{code}
\end{block}

\section{Operator Precedence}
Associativity: (L)eft, (R)ight, (N)on, (C)hained, (X) list
\begin{block}
\begin{tabularx}{\hsize}{@{}p{1ex}p{2.2cm}X@{}}
  L & Method postfix   & \Code{.meth .+ .? .* .() .[] {.\{\}} .{\<}{\>} {.\string«\string»} .:: .= {.\^} .:}\\
  N & Autoincrement    & \Code{++ --}\\
  R & Exponentiation   & \Code{**}\\
  L & Symbolic unary   & \Code{! + - {\~} ? | || {+\^} {\~\^} {?\^} \^}\\
  L & Multiplicative   & \Code{* / {\%} {\%\%} {+\&} {+\<} {+\>} {\~\&} {\~\<} {\~\>} {?\&} div mod gcd lcm}\\
  L & Additive         & \Code{+ - +| {+\^} {\~|} {\~\^} ?| ?\^}\\
  L & Replication      & \Code{x xx}\\
  X & Concatenation    & \Code{\~}\\
  X & Junctive and     & \Code{\&}\\
  X & Junctive or      & \Code{| \^}\\
  L & Named unary      & \Code{temp let}\\
  N & Structural infix & \Code{but does {\<}={\>} leq cmp .. ..{\^} {\^}..}\\
  C & Chaining infix   & \Code{!= == {\<} {\<}= {\>} {\>}= eq ne lt le gt ge {\~}{\~} === eqv !eqv}\\
  X & Tight and        & \Code{\&\&}\\
  X & Tight or         & \Code{|| {\^}{\^} // min max}\\
  R & Conditional      & \Code{?? !! ff fff}\\
  R & Item assignment  & \Code{= {=}{\>} += -= **= xx= .=}\\
  L & Loose unary      & \Code{so not}\\
  X & Comma operator   & \Code{, :}\\
  X & List infix       & \Code{Z minmax X {X\~} X* Xeqv}\\
  R & List prefix      & \Code{print push say [+] [*] any Z=}\\
  X & Loose and        & \Code{and andthen}\\
  X & Loose or         & \Code{or xor orelse}\\
  X & Sequencer        & \Code{{\<}== {==\>} {\<}{\<}== =={\>}{\>}}\\
  N & Terminator       & \Code{;\quad {\{\dots\}}\quad ) ] \}\quad unless}
\end{tabularx}
\end{block}

\section{Types}
\begin{block}
\lr{Mu, Any, Cool}{root type, base class, dual string/num}
\lr{Bool, Bit}
\lr{Numeric, Int, Rat, FatRat, UInt, Num, Complex}
\lr{Stringy, Str, Blob, Buf}
\lr{Char, Byte, Codepoint, Grapheme}
\lr{int32, complex64, buf8, buf32, \dots}{Native types}
% utf8 IO Junction
% Supplier Supply Whatever Match Nil Parcel Capture Signature Pair Range Set Bag Mix
% SetHash BagHash MixHash Scalar Array Hash Code Enum Order TrigBase Block
% Routine Sub Method Regex Cursor Failure Exception Instant Duration Date
% DateTime
\lr{subset <Pos> of Numeric where * {\>} 0}{custom}
\end{block}

\section{Str}
\begin{block}
\lr{.chars, .codes}{num chars (graphemes), code points}
\lr{.chop(\$chars=1), .chomp}
\lr{.lc, .uc, .fc, .tc, .tclc}{lower, upper, fold, title}
% \lr{.wordcase(:\&filter=\&tclc, :\$where=True)}{apply filter to each word matching where}
% \lr*{.wordcase(:\&filter=\&tclc, :\$where=True)}{".words {--\>} grep \$where {--\>} map \&filter"}
\lr{.unival, .univals}{numeric value ("'¾'.unival" == 0.75)}
\lr{.encode(\$encoding=\$?ENC,\$nf=\$?NF)}
\lr{.index(\$needle, \$startpos=0), .rindex(\dots)}{Int | undef}
\lr{.split(<Str|List|Regex>, \$limit=Inf, :\$skip-empty, \dots)}
\lr{.comb(<Str|Regex|Int>, \$limit=Inf)}{list of matches}
\end{block}

\section{Numerics}
\begin{block}
\lr{:3\<102\>, 199.base(3)}{from/to base 3}
\lr{\$pi = 3.14e0}{Force floating-point}
\end{block}

\section{Subs / Callables / \dots}
\begin{block}
\lr{.assuming()}{Currying}
\lr{.arity, .count, .signature}{min/max positnl arity, sig}
\lr{is export}{exported from module}
\lr{is cached}{EXPERIMENTAL: memoization}
\lr{is pure}{pure function compiler hint}
\lr{is DEPRECATED(Str \$msg)}{generate warning on use}
\lr{is rw}{can be used as lvalue}
\lr{is hidden-from-backtrace}{}
\end{block}

\section{Processes}
\begin{block}
\lr{shell(\$<cmd>), run(*@<args>)}{command with/out shell return "Proc" objects}
\lr{qx[ <cmd> ], qqx[ <cmd> ]}{execute and canpture STDOUT}
\end{block}
\begin{block}[Proc]
\lr{.exitcode, .signal}{Exit code or -1; signal or 0}
\lr{.pid, .command}{PID; command as array}
\lr{.sink}{OK or throw "X::Proc::Unsuccessful"}
\begin{code}
  sub run( :\$in = '-', :\$out = '-', :\$err = '-',
    :\$chomp = True, :\$bin = False, :\$merge = False,
    Str:D :\$enc = 'UTF-8', Str:D \$nl = "{\bs}n",
    :\$cwd = \$*CWD, Hash() :\$env = \%*ENV,
    *@args
  ) returns Proc:D
\end{code}
\begin{code}
  \$proc := run 'echo', 'Hallo world', :out;
  my \$output := \$proc.out.slurp-rest;
  my \$p1 := run 'echo', 'Hello, world', :out;
  my \$p2 := run 'cat', '-n', :in(\$p1.out), :out;
  say \$p2.out.get;
\end{code}
\end{block}
\begin{block}[Proc::Async]
\begin{code}
  my \$proc := Proc::Async.new('echo', 'Hello World');
  \$proc.stdout.tap(-> \$v \{ print "Output: \$v" \});
  \$proc.stderr.tap(-> \$v \{ print "Error:  \$v" \});
  my \$promise := \$proc.start;
  await \$promise; # Optionally wait for completion
\end{code}
\end{block}

\section{Pod}
\begin{block}
\begin{code}
  =begin comment
  Multi line comments...
  =end comment
\end{code}
\end{block}

\section{Testing}
\begin{block}
Execute using "prove --exec perl6 myfile.t"
\begin{code}
  use Test;
  pass "«Hooray»"; flunk "«Bummer»"; diag "«A Message»";

  ok   \$«val», \$«msg»;
  nok  \$«val», \$«msg»;
  is   \$«val», «A::Type», \$«msg»;  \comment{# === comparison}
  is   \$«val», \$«wanted», \$«msg»; \comment{# string eq}
  isnt \$«val», \$«wanted», \$«msg»;
  like \$«val», \$«pat», \$«msg»;    \,\comment{# regex, also unlike()}

  cmp-ok \$«val», '«==»', \$«wanted», \$«msg»; \comment{# any operator}
  is-deeply \$«val», \$«wanted», \$«msg»;
  is-approx \$«val», \$«wanted», \$«msg»;    \comment{# within 1e-5}
  isa-ok \$«val», \$type, \$«msg»;   \,\comment{# type name or class}

  subtest \{ «\dots tests \dots» \}, \$«msg»; \comment{# test group}
  dies-ok \{ «\dots code \dots» \}, \$«msg»;
  lives-ok \{ «\dots code \dots» \}, \$«msg»;
  throws-like \{ «\dots code \dots» \}, «X::Type», \$«msg»;

  done;
\end{code}
\end{block}

\section{Objects}
\begin{block}
\begin{code}
  role «MyRole» \{ ... \}
  class «MyParent» \{ ... \}

  class «Foo» is «MyParent» does «MyRole» \{
    my \$static      = True; \comment{# private static}
    my \$.pub-static = 42;   \comment{# public static}

    has \$!x;       \comment{# private}
    has \$.y;       \comment{# public read-only}
    has \$.z is rw; \comment{# public read-write}
    has @.w;       \comment{# public read-only array}

    sub «foo»() \{ ... \}  \comment{# private sub}

    ## UNTESTED: static only
    method «blah»(Foo:U \$type:) \{ ... \}

    method «add-item»(\$val) \{
      self.«whatever»;   \comment{# Methods on \texttt{self}}
      @!w.push: \$val;  \comment{# Note! \texttt{@!w}}
    \}

    \comment{# private method, call with self!my-private}
    method !«my-private»(...) \{ ... \}

    \comment{# We get a key/value .new() for free.}
    \comment{# Create multi-method for positional}
    multi method new(\$x, *@w) \{
      return self.bless(:\$x, :w[42, |@w]);
    \}

    \comment{# Run code at build time}
    method BUILDALL (|c) \{
      nextsame;
      \comment{# other code, self already populated...}
    \}

    \comment{# Manipulate your attrs at init}
    submethod BUILD (\$!y, \$!z, @w) \{
      \comment{# auto-bind y, z (note "!"), manually set w}
      @!w = [ 42, |@w ];
    \}
  \}
\end{code}
\end{block}
\begin{block}[Introspection]
\lr{\$o.WHAT, \$o {\~\~} Foo, \$o {\~\~} MyRole}{obj class, type tests}
\lr{\$o.\^attributes, \$o.\^methods, \$o.\^parents}
\end{block}


\begin{block}[``Magic'' Methods]
\lr{method CALL-ME(|arguments)}{does Callable}
\end{block}

\section{Enum}
\begin{block}
Value types may be Numerical or Str with Int as default.
\begin{code}
  enum Foo ( ONE => 1, TWO => 2 );
  enum Names ( FIRST => "First", LAST => "Last" );
  enum Bar << A B C :D(42) >>;
  say ONE, \": \", ONE.value;
  for Bar.enums \{ say \$\^x.key, \": \", \$\^x.value \}
\end{code}
\end{block}

\section{Grammars}
\begin{block}
\begin{code}
  class Action \{
    has \$.type;   has \$.args;
    method act(\$c) \{ ... \}
  \}
\end{code}
\end{block}
\begin{block}
\begin{code}
  grammar MyGrammar \{
    token TOP \{ ^ <action> [ {\bs}n+ <action> ]* {\bs}s* \$ \}
    token ws \{ {\bs}h* \}

    proto rule action \{*\}
    rule action:sym<*> \{
      <type=.identifier> <args=.rest>
    \}
    rule action:sym<whatever> \{ ... \}

    token identifier \{ {\bs}w+ \}
    token rest       \{ .*  \}
  \}
\end{code}
\end{block}
\begin{block}
\begin{code}
  class MyActions \{
    method TOP(\$/) \{ \$/.make: \$<block>».made \}

    method action:sym<*>(\$/) \{
      \$/.make: Action.new(
          type => ~\$<type>, args => ~\$<args>
      );
    \}
  \}
\end{code}
\end{block}
For precedence / calculator:
\begin{block}
\begin{code}
  rule expr \{ <expr> [ <infix> <expr> ]+ \}

  proto token infix \{*\}
  token infix:sym<+>                \{ <sym> \}
  token infix:sym<-> is equiv:<+>   \{ <sym> \}
  token infix:sym<*> is tighter:<+> \{ <sym> \}
  token infix:sym</> is equiv:<*>   \{ <sym> \}
  token infix:sym<^> is tighter:<*> \{ <sym> \}
\end{code}
\end{block}


\section{Core Classes}
\begin{block}[DateTime]
\lr{DateTime.new(:\$year, :\$month, \dots, \$timezone)}
\lr{DateTime.now; DateTime.new(<Instant>)}
\lr{.clone(:\$year, :\$month, \dots)}{copy then update}
\lr{.hour, .minute, .whole-second, .second}
\lr{.year, .month, .day}{natural ints}
\lr{.timezone := .offset}{seconds offset from UTC}
\lr{.day-of-week, .day-of-year, .days-in-month}
\lr{(\$year, \$week) = .week}{\$year may differ from .year}
\lr{.instant, .posix}
\lr{.later(<year> {=\>} <1>).earlier(<day> {=\>} <1>)}{one attr at a time}
\lr{.Date, .truncated-to(\"<day>\")}{As a "Date" or truncated}
\lr{.utc, .local, .in-timezone(<tz>)}
\end{block}

\begin{block}[IO::Path]
\lr*{.dirname, .basename, .extension, .volume}{``foo/bar.baz'' "--\>" ``foo'', ``bar.baz'', ``baz''}
\lr{.parts}{Hash with "dirname", "path", and "volume"}
\lr{.watch}{"Supply" notifying of changes}
\lr{.absolute, .relative, .is-absolute, .is-relative}
\lr{.dir, .dir(test ={\>} *.d)}{readdir, see ``Files'' above}
\lr{chmod(<mode>,<paths>), mkdir(<mode>,<paths>)}{modes: 0o777}
\lr{move(<f>,<t>), rename(<f>,<t>), copy(<f>,<t>)}{take ":createonly"}
\lr{link(<old>,<new>), symlink(<o>, <n>)}
\lr{unlink(<files>), rmdir(<dirs>)}
\lr{.modified, .accessed, .changed}{"Instant" obj}
\end{block}

\begin{block}[IO::Handle]
\lr{.get, .getc}{one line, character}
\lr{.lines, .words}{lazy list of lines or words}
\lr{.print}{general output}
\lr{.read, .write}{read/write binary "Blob" data}
\lr{.seek(\$offset, \$whence), .tell}
\lr{.ins}{number of lines read so far}
\lr{.slurp-rest(:bin | :enc\<...\>)}
\subtitle{From \texttt{IO} base}
\lr{.say, .note, .dd}{println, print gists, print data}
\end{block}

\begin{block}[Proc::Status]
\lr{?\$s}{true on successful exit}
\lr{+\$s, \$s.exit()\rv{Int}}{exit status}
\lr{.signal()\rv{Int}}{Signal::SIGSEGV, Signal::SIGPIPE, \dots}
\lr{.pid()\rv{Int}}{undef if unknown}
\end{block}

% \section{Rakudo Star Classes}
% \section{Non-Core Classes}

\section{Various Examples}

\begin{block}[Method cascade]
\begin{code}
  \comment{\# in block \$\_ is \$obj so methods called on \$obj}
  my \$obj := My::Class.new(title {=\>} "abc") andthen \{
    .foo();
    .bar();
    .baz();
  \}
\end{code}
\end{block}

\begin{block}[Operator Overloading]
\begin{code}
  sub infix:\< ?= \> (\$l is rw, \$r) \{
    \$l = \$r if defined \$r
  \}
  my \$new = 15; my \$var;
  \$new ?= \$var; # No assignment yet

  sub circumfix:\<{\unicode ⟅} {\unicode ⟆}\> ( *@array ) \{
      for @array.keys -\> \$i \{
          @array[\$i] = @array[\$i].flip;
      \}
      @array;
  \}
  say {\unicode ⟅}'cat', 'dog', 'bird'\!\!{\unicode ⟆}; #> [tac god drib]
\end{code}
\end{block}

\end{multicols*}
\end{document}
